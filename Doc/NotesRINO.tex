\documentclass{llncs}


\usepackage{booktabs} % For formal tables
\usepackage{mathtools} % extension and fixes of/in amsmath
\usepackage{amsmath,amssymb,tikz}
\usepackage{epsfig}
\usepackage[]{algorithm2e}

\DeclareFontEncoding{LGR}{}{}
\DeclareSymbolFont{sfgreek}{LGR}{cmss}{m}{n}
\SetSymbolFont{sfgreek}{bold}{LGR}{cmss}{bx}{n}
\DeclareMathSymbol{\sfdelta}{\mathord}{sfgreek}{`d}

\DeclareMathSymbol{\sftau}{\mathord}{sfgreek}{`t}

\usepackage{upgreek}
%\usepackage[ruled]{algorithm}

% graphics
\usepackage{graphicx}
\DeclarePairedDelimiter{\parentheses}{(}{)}

\newcommand{\lenvelope}{\mathopen{\mathrel{[}\joinrel\mathrel{[}}}
\newcommand{\renvelope}{\mathclose{\mathrel{]}\joinrel\mathrel{]}}}

%\usetikzlibrary{arrows.meta, backgrounds}

\tikzset{%
    curve/.style={thick},
    deriv/.style={thick,dashed},
    % term/.style={thick},
    termderiv/.style={thick,dashdotted},
    leftpoint/.style={color=black},
    rightpoint/.style={color=white, draw=black},
    leftp/.style={
        {Circle[width=4,length=4]}-,
        shorten <=-2},
    rightp/.style={
        -{Circle[width=4,length=4,fill=white]},
        shorten >=-2},
    cadlag/.style={
        {Circle[width=4,length=4]}-{Circle[width=4,length=4,fill=white]},
        shorten >=-2,
        shorten <=-2
    },
	rights/.style={
        -{Turned Square[width=4,length=4]},
        shorten >=-2}
}

% definition equality
\newcommand{\defeq}{\mathrel{\overset{\makebox[0pt]{\mbox{\normalfont\tiny def}}}{=}}}

% maths quantifiers
\newcommand{\mexists}[1]{\exists\mkern2mu #1}
\newcommand{\mforall}[1]{\forall\mkern2mu #1}
\newcommand{\holds}{\mathrel{\colon}}

\newcommand{\mimply}{\Rightarrow}
\newcommand{\mbisubjunct}{\Leftrightarrow}
\newcommand{\tbisubjunct}{\quad\text{iff}\quad}
\DeclarePairedDelimiterX\Set[1]{\lbrace}{\rbrace}{\def\with{\;\delimsize\vert\;}#1}

% set complement
\newcommand*{\scomplement}[1]{#1^\complement}

% compact interval [#1,#2]
\newcommand{\compactum}[2]{[#1,#2]}
% open interval
\newcommand{\open}[2]{(#1,#2)}
% left closed, right open interval
\newcommand{\closedopen}[2]{[#1,#2)}

% functions: f\from\R\to\R
\newcommand*{\from}{\colon}

% partition of an interval
\newcommand{\partition}[3][]{
	\Set{#2<\ifthenelse{\equal{#1}{}}{}{#1<}\ldots<#3}}

% converge down to
\newcommand{\downto}{\searrow}

% converge up to
\newcommand{\upto}{\nearrow}

% Differential (upface d)
\DeclareMathOperator{\dif}{d \!}

\newcommand{\R}{\mathbb{R}}
\newcommand{\N}{\mathbb{N}}
\def\Z{{\mathbb Z}}
\def\bfm#1{\protect{\makebox{\boldmath $#1$}}}
\def\z {\bfm{z}}
\def\K{{\Bbb I \Bbb K}}
\def\x {\bfm{x}}
\def\bbr{{\Bbb R}}
\newcommand{\pro}{\mbox{pro }}
\def\x{\bfm{x}}
\def\J{\bfm{J}}
\def\f{\bfm{f}}
\def\I{{\Bbb I \Bbb R}}
\DeclareMathOperator{\range}{range}
\def\y {\bfm{y}}
\newcommand{\dual}{\mbox{dual }}
\def\f {\bfm{f}}


\def\r {\bfm{r}}


\newcommand{\D}[2][1]{\Dprime@cases#1!{#2}}
\def\Dprime@cases{\@ifnextchar{!}{\Dprime@prime}{\@ifnextchar({\Dprime@counted}{\Dprime@prime}}}
\def\Dprime@counted(#1)!#2{{#2}^{(#1)}}
\def\Dprime@prime#1!#2{%
\ifthenelse{\equal{#1}{0}}{{#2}}
  {\ifthenelse{\equal{#1}{1}}{#2'}
    {\ifthenelse{\equal{#1}{2}}{#2''}
      {\ifthenelse{\equal{#1}{3}}{#2'''}
        {%% general case
          {#2}^{(#1)}%
        }
      }
    }
  }}
% right derivative
\newcommand{\Dright}[2][1]{\D[#1]{#2_+}}

% left derivative
\newcommand{\Dleft}[2][1]{\D[#1]{#2_-}}

% derivative with Leibniz notation d/dx f(x)
\newcommand{\dd}[2]{\frac{\dif #1}{\dif #2}}

% partial derivative
\newcommand{\DD}[2]{\frac{\partial #1}{\partial #2}}

% differential in integral
\newcommand{\dx}[1][x]{\dif #1}
% integral
\newcommand{\integral}[2]{\int_{#1}^{#2}}
\newcommand{\denseintegral}[2]{\int_{#1}^{#2}\!\!}

% piecewise/cadlag differentiable functions
\newcommand{\cadlag}{\foreignlanguage{frenchb}{c\`adl\`ag}\xspace}
\newcommand{\Cadlag}{\foreignlanguage{frenchb}{C\`adl\`ag}\xspace}
\newcommand{\Cnpw}[3][]{C_{\mkern-2mu\mathrm{pw}}^{#1}\ifthenelse{\equal{#2}{}}{}{\ifthenelse{\equal{#3}{}}{(#2)}{(#2,#3)}}}

% norms
\DeclarePairedDelimiter{\abs}{\lvert}{\rvert}
\DeclarePairedDelimiter{\nnorm}{\lVert}{\rVert}
\DeclarePairedDelimiter{\supnorm}{\lVert}{\rVert_{\sup}}

% right hand side of DDEs
\newcommand{\frhs}{f\big(t,x(t),x(t-\tau_1),\ldots,x(t-\tau_k)\big)}

% definition domain of right hand side
\newcommand{\deff}{\R\times\R^n\times\ldots\times\R^n}
\newcommand{\tzero}{\sigma}
\newcommand{\taumax}{\tau_{\max}}
\newcommand{\taumin}{\tau_{\min}}

% first-order logic of real arithmetic
\newcommand{\FOLR}{\textnormal{FOL}$_\R$\xspace}
\newcommand{\asfmlfolR}{\chi}
\newcommand{\bsfmlfolR}{\varphi}

% differential dynamic logic
\newcommand{\dL}[1][]
{\text{\upshape\textsf{d{\kern-0.1em}$\mathcal{L}$}}\xspace}

% hybrid programs
\newcommand{\HP}{\text{HP}\xspace}%
\newcommand{\HPs}{\text{HPs}\xspace}%

% delay hybrid programs
\newcommand{\dHP}{$\updelta${\kern-0.03em}\text{HP}\xspace}
\newcommand{\dHPs}{$\updelta${\kern-0.03em}\text{HPs}\xspace}

% schema variables for programs
\newcommand{\asprg}{\alpha}
\newcommand{\bsprg}{\beta}
\newcommand{\csprg}{\gamma}

\newcommand{\init}{\text{init}}
\newcommand{\ctrl}{\text{ctrl}}
\newcommand{\plant}{\text{plant}}
\newcommand{\req}{\text{req}}
\newcommand{\Controller}{\text{DiscCtrl}()}

\newcommand{\assign}{\mathrel{{:}{=}}}
\newcommand{\hseq}{;}
\newcommand*{\htest}[1]{\ensuremath{?#1}}
\newcommand*{\hchoice}[2]{\ensuremath{{#1}\cup{#2}}}
\newcommand*{\hrepeat}[2][*]{\ensuremath{{#2}^{#1}}}
\newcommand{\hevolvein}[2]{\ensuremath{{\hevolve{#1}}\,\&\,#2}}
\newcommand{\syssep}{,}
\newcommand{\humod}[2]{#1\hspace{-0.05em}\assign\hspace{-0.07em}#2}
\let\Dupdate\hupdate
\let\Dumod\humod
\let\pevolve\hevolve
\let\pevolvein\hevolvein

% delay differential dynamic logic
\newcommand{\DDEdL}{\textsf{DDEdL}}
%{$\sfdelta${\kern-0.03em}\textsf{d{\kern-0.1em}$\mathcal{L}$}\xspace}

\newcommand*{\allvars}{\mathcal{V}}
\newcommand*{\constants}[1][]{\mathcal{C}_{#1}}
\newcommand*{\diffvars}{\D{\allvars}}
\newcommand*{\delayvars}[1][]{\allvars[\constants[#1]]}
\newcommand*{\delaydiffvars}[1][]{\diffvars[\constants[#1]]}

\newcommand{\mult}{\cdot}

\newcommand{\Dx}[1][]{\D{x}[#1]}

% \der{#1} differential of #1 for differential invariant operators
\newcommand{\der}[1]{(#1)'}

\newcommand{\hs}[2][-T]{\lforall{_s[#1)}{#2}}
\newcommand{\hsc}[2][-T]{\lforall{_s[#1]}{#2}}

\newcommand{\delayinterval}[1][-T]{\closedopen{#1}{0}}
\newcommand{\closeddelayinterval}[1][-T]{\compactum{#1}{0}}
\newcommand{\statespace}[1][-T]{\Cnpw[1]{\closeddelayinterval[#1]}{\R^n}}

% schema variables for terms
\newcommand{\astrm}{\theta}
\newcommand{\bstrm}{\eta}
\newcommand{\istrm}[1]{\theta_{#1}}

% schema variables for formulas
\newcommand{\asfml}{\phi}
\newcommand{\bsfml}{\psi}

% states
\newcommand{\states}{\mathcal{S}}
\newcommand{\asstate}{\nu}
\newcommand{\bsstate}{\omega}
\newcommand{\csstate}{\mu}

% dynamic semantics
% terms: ivaluation
% formula: imodel
% programms: ireachability
\newcommand{\ireachability}[1]{\iget[access]{#1}\parentheses}

\newcommand{\trajectory}{\gamma}
\newcommand{\past}{r}
\newcommand{\duration}{R}

% \dbox variant for \index
\newcommand{\dibox}[1]{[#1]}
% \ddiamond variant for \index
\newcommand{\didia}[1]{\protect\langle#1\protect\rangle}

% static semantics
% use \freevars*{} for parentheses autostretch, \freevars[\big]{} for manual
\newcommand{\freevars}{\operatorname{FV}\parentheses}
\newcommand{\boundvars}{\operatorname{BV}\parentheses}
\newcommand{\mustboundvars}{\operatorname{MBV}\parentheses}

% History Horizon
\newcommand{\HH}{\operatorname{HH}\parentheses}

% evolution domain constraint
\newcommand{\ivr}{\asfmlfolR}

% differential invariant
\newcommand{\inv}{\bsfmlfolR}

% load one single symbol from stix font for steps axiom

% Settings %%%%%%%%%%%%%%%%%%%%%%%%%%%%%%%%%%%%%%%%%%%%%%%%%%%%%%%%%%%%%%%

 \newcommand\ForAuthors[1]%          %  temporary remark for the
 {\par\smallskip                     %  authors:
  \begin{center}%                    %
   \fbox%                            %    --------
   {\parbox{0.9\linewidth}%          %    |  #1  |
    {\raggedright\sc--- #1}%         %    --------
   }%                                %
  \end{center}%                      %
  \par\smallskip                     %
 }        

%\usepackage[ruled]{algorithm2e} % For algorithms


\begin{document}

\title{RINO: some implementation notes}

\author{Sylvie Putot}


\maketitle

\section{ODEs}
We are given a system of ODEs
\[ \dot z(t) = f(z,u,t) \]
with $z(0)=z^0$.
Here, for efficiency reasons, we do not systematically rewrite $u$ as a dimension of the system.

\begin{itemize}
\item sysdim is the dimension of the state space of the system of ODE
\item sysdim\_params is the dimension of the (possibly uncertain) parameters $u$ that are not dimensions of the Jacobian
\item inputsdim is the dimension of the uncertain parameters and inputs $u$ that appear in the Jacobian
\item when these parameters can be piecewise constant and not constant, fullinputsdim is the full dimension of these uncertain parameters and inputs
\item jacdim = sysdim + fullinputsdim
\end{itemize}

\texttt{TM\_Jac::build} and TM\_Jac::eval\_Jac build and evaluate the Taylor model for the Jacobian $J_{ij}(t,\z_0)=\frac{\partial z_i}{\partial z_{0,j}}(t,\z_0)$ and $J_{ij}(t,\z_0)=\frac{\partial z_i}{\partial u_{0,j}}(t,\z_0)$,  as defined in Equation (17) of [HSCC2017]: 
\begin{itemize}
\item in class TM\_Jac, J corresponds to $J_j$ in (17), and J\_rough to  $R_{j+1}$.
\item in TM\_Jac::eval\_Jac, Jaci corresponds to $Jac_zf^{[i]}$ in (17)
\end{itemize}
The code is slightly more difficult to understand when jacdim is not equal to sysdim. Let us consider an example with dimension 2 and 2 parameters, the sensitivity matrix J we want to compute relies on products \[ \mbox{Jac}_zf^{[i]}([{\z}_{j}]). [{\J}_j] \] of matrices of dimensions dimension (jacdim $\times$ jacdim) with, for example for $i=1$,
\begin{equation*}
\mbox{Jac}_zf^{[1]}([{\z}_{j}]) . [{\J}_j]  = 
\begin{pmatrix}
\frac{\partial f_0}{\partial z_0} &  \frac{\partial f_0}{\partial z_1}   & \frac{\partial f_0}{\partial u_0} & \frac{\partial f_0}{\partial u_1} \\
\frac{\partial f_1}{\partial z_0} &  \frac{\partial f_1}{\partial z_1}   & \frac{\partial f_1}{\partial u_0} & \frac{\partial f_1}{\partial u_1} \\
0  & 0  & 0 & 0  \\
0 & 0 & 0 & 0
\end{pmatrix}
.
\begin{pmatrix}
\frac{\partial z_0}{\partial z^0_0} &  \frac{\partial z_0}{\partial z^0_1}   & \frac{\partial z_0}{\partial u_0} & \frac{\partial z_0}{\partial u_1} \\
\frac{\partial z_1}{\partial z^0_0} &  \frac{\partial z_1}{\partial z^0_1}   & \frac{\partial z_1}{\partial u_0} & \frac{\partial z_1}{\partial u_1} \\
0  & 0  & 1 & 0  \\
0 & 0 & 0 & 1
\end{pmatrix}
\end{equation*} 
%Indeed, if the parameters $u$ are constant

Thus, only part of the matrices are relevant. For efficiency, we store only the submatrices of dimension (sysdim $\times$ jacdim), and adapt the matrix operations such as matrix multiplication (multJacfzJaczz0) in that respect. We thus write
\begin{equation*}
\mbox{Jac}_zf . [{\J}_j]  = \begin{pmatrix} \mbox{Jac}_zf. \mbox{Jac}_{z^0}z &   \mbox{Jac}_zf \mbox{Jac}_{u}z + Jac_u^f
\end{pmatrix}
\end{equation*} 
This actually corresponds to Equations (8) and (9) (where beta is z0) of [HSCC2019].

\section{Examples}

\paragraph{Piecewise constant inputs}
Examples 19 and 21 are the following system:
\[ \dot x(t) = f(u,t) = 2+2u + (1-2u)t \]
with $u \in [0,1]$ and $x(0)=0$. For now, time is embedded as a component of the system. 

This example is a good example to test piecewise constant time-varying inputs. Let us take time horizon of 2 and stepsize of 1. 
If $u$ is constant equal to $u_0$ on the whole interval, then 
\[ x(t) = \int_0^t 2+2u_0 + (1-2u_0)s ds  = [ (2+2u_0)s+(1-2u_0)s^2/2]_0^t \]
and  in particular $x(2) = 6$.
If $u$ is only piecewise constant on each time setp of size 1, then 
\[ x(2) = \int_0^1 2+2u_1 + (1-2u_1)s ds + \int_1^2 2+2u_2 + (1-2u_2)s ds = 6 + u_1 - u_2\]
and if both $u_1$ and $u_2$ are in  $[0,1]$ then  $x(2) \in [5,7]$. 

On thix example, the maximal outer and inner approximations are exact. I thus added a non linearity to get non exact approximations in Example 22 which encodes \[ \dot x(t) = f(u,t) = 2+2(u^2+u) + (1-2(u^2+u))t \] with $u \in [0,1]$ and $x(0)=0$. 

\bibliographystyle{plain}
\bibliography{NotesRINO}
%\newpage
%\appendix
%\section{Details of benchmarks}
%\input{appendix.tex}

\end{document}


